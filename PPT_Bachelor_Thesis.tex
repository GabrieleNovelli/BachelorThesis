\documentclass{beamer}
\usetheme{Berkeley}

\usepackage[italian]{babel}
\usepackage{newlfont}
\usepackage{color}
\usepackage{float}
\usepackage{frontespizio}
\usepackage{amsmath,amssymb}
\usepackage{amsthm}
\usepackage{geometry}
\usepackage{tikz}
\usepackage{biblatex}
\usepackage{csquotes}
\usepackage{pgfplots}
\usepackage{hyperref}
\usepackage{amssymb}
\usepackage{comment}

\theoremstyle{definition}
\newtheorem{Def}{Definizione}

\theoremstyle{Theorem}
\newtheorem{Theo}[Def]{Teorema}
\newtheorem{Prop}[Def]{Proposizione}

\newtheorem{Lm}[Def]{Lemma}

\theoremstyle{definition}
\newtheorem{Ex}[Def]{Esempio}

\theoremstyle{definition}
\newtheorem{Lem}[Def]{Lemma:}

\theoremstyle{definition}
\newtheorem{Obs}[Def]{Osservazione:}


\begin{document}
\begin{frame}
	\title{Gruppi di Lie e rappresentazioni: la quantizzazione del momento angolare}
	%\author{Gabriele Novelli, Relatore: Prof. Rita Fioresi}
    \date{  }
    \institute[VFU]
    {
    	Scuola di Scienze\\
    	Dipartimento di Fisica e Astronomia\\
    	Corso di Laurea in Fisica
    }
    \maketitle
    	 Presentata da: \hspace{4.5cm} Relatore: \\
    	 Gabriele Novelli \hspace{4.5cm} Prof. Rita Fioresi
\end{frame}
\begin{frame}{Varietà Differenziabili, spazio tangente e campi vettoriali}
	\begin{itemize}
		\item Varietà differenziabile: una varietà topologica munita di un atlante.
	\end{itemize}
	\begin{exampleblock}{Esempio}
		L'insieme $GL_n(\mathbb{R})=\{A\in M_{n\times n}|det(A)\neq0\}$ è una varietà differenziabile.
	\end{exampleblock}
	\begin{itemize}
	\item L'insieme di tutte le derivazioni in $p$ è lo spazio tangente $T_pM$.
	\item Campo vettoriale: $X: p\rightarrow X_p\in T_pM$.
\end{itemize}
	Base per lo spazio tangente: ${\partial\over \partial x^i}\bigg\rvert_p(f)={\partial\over \partial r^i}\bigg\rvert_{\phi(p)}(f\circ \phi^{-1})$ dove $r^i$ sono le coordinate di $\mathbb{R}^n$.	
\end{frame}

\begin{comment}
\begin{frame}{Varietà Differenziabili}
\begin{Def}
Carta centrata in $p\in M$: $(U,\phi)$, $U$ aperto contenente $p$, $\phi:M\rightarrow\mathbb{R}^n$ omeomorfismo.
%$\mathbb{U}=\{(U_i,\phi_{i})\}$ collezione di carte compatibili su $M$ tali che $M=\bigcup_i U_i$ è detta \textit{atlante}.%
Una varietà topologica munita di un atlante è detta \textit{varietà differenziabile}.
\end{Def}
\begin{exampleblock}{Esempio}
L'insieme $GL_n(\mathbb{R})=\{A\in M_{n\times n}|det(A)\neq0\}$ è una varietà differenziabile.
\end{exampleblock}
\begin{center}
\includegraphics[scale=0.4]{Carta.jpg}
\end{center}
\end{frame}
\begin{frame} {Spazio tangente e campi vettoriali}
\begin{Def}
Derivazione o vettore tangente in p: $D_p:C^\infty_p(M)\rightarrow\mathbb{R}$ tale che $D_p(fg)=D_p(f)g(p)+f(p)D_p(g)$. L'insieme di tutte le derivazioni in $p$ è lo spazio tangente $T_pM$.
\end{Def}
Base per lo spazio tangente: ${\partial\over \partial x^i}\bigg\rvert_p(f)={\partial\over \partial r^i}\bigg\rvert_{\phi(p)}(f\circ \phi^{-1})$ dove $r^i$ sono le coordinate di $\mathbb{R}^n$.
\begin{Def}
Campo vettoriale: $X: p\rightarrow X_p\in T_pM$.
\end{Def}

\end{frame}
\end{comment}

\begin{frame} {Gruppi di Lie e rivestimenti}
	\begin{Def}
		Un \textit{gruppo di Lie} $G$ è una varietà differenziabile con la struttura di gruppo, tale che le operazioni di moltiplicazione $\cdot:G\times G\rightarrow G$ e inversione $i:G\rightarrow G$ sono $C^\infty$.
	\end{Def}
\begin{itemize}
	\item $GL_n(\mathbb{R})$, $SL_n(\mathbb{R})$, $SO_3(\mathbb{R})$ e $SU_2(\mathbb{C})$ sono gruppi di Lie.
\end{itemize}\vspace{0.5cm}
Azione di gruppo: $a:G\times X\rightarrow X$ tale che
\begin{itemize}
	\item $a(gh,x)=a(g,(a(h,x)))$ $\forall$ $g,h\in G$; 
	\item $x\in X$; $a(e,x)=x$ $\forall x\in X$.
\end{itemize}
\end{frame}
\begin{frame} {Gruppi di Lie e rivestimenti}
	\begin{Def}
		Dati due spazi topologici $X$ e $\tilde{X}$ ed una mappa continua $p:\tilde{X}\rightarrow X$, allora $p$ è detta \textit{rivestimento} se:
		\begin{itemize}
			\item $p$ è suriettiva;
			\item per ogni $x\in X$ esiste $U\subset X$ intorno di $x$ tale che $p^{-1}(U)=\bigsqcup_j V_j$. Dove $\{V_j\}$ è una collezione di sottoinsiemi aperti di $\tilde{X}$ disgiunti a due a due e tale che, per ogni $j$, $p\rvert_{V_j}:V_j\rightarrow U$ è omeomorfismo. 
		\end{itemize}
	\end{Def}
\vspace{0.5 cm}
\begin{itemize}
	\item $p$ è detta rivestimento universale se $\tilde{X}$ è semplicemente connesso.
\end{itemize}
\end{frame}
\begin{frame} {Algebre di Lie}
	\begin{Def}
		Si dice \textit{algebra di Lie} uno spazio vettoriale $\mathfrak{g}$ in cui è definita un'operazione binaria bilineare (detta Lie bracket) $[,]:\mathfrak{g}\times\mathfrak{g}\rightarrow\mathfrak{g}$ tale che verifica:
	\begin{itemize}
		\item Antisimmetria: $[x,y]=-[y,x]$, $\forall x,y\in\mathfrak{g}$;
		\item Identità di Jacobi: $[x,[y,z]]+[y,[z,x]]+[z,[x,y]]=0$, $\forall x,y,z\in \mathfrak{g}$ 
	\end{itemize}
\end{Def}
\begin{Def}
	Se $G$ è un gruppo di Lie, $X$ campo vettoriale è invariante a sinistra se $dl_g(X)=X$.
\end{Def}
\end{frame}
\begin{frame} {Algebre di Lie}
\begin{Def}
	Chiamiamo $Lie(G)=\mathfrak{g}$ algebra di Lie di $G$, l'insieme di tutti i campi vettoriali invarianti a sinistra su di un gruppo di Lie $G$.
\end{Def}
\begin{alertblock}{}
	E' possibile dimostrare che $Lie(G)\cong T_eG$.
\end{alertblock}
Alcune algebre di Lie rilevanti:
\begin{itemize}
	\item $\mathfrak{gl}(V)\cong End(V)$ 
	\item $\mathfrak{sl_n(\mathbb{R})}=\{A\in M_{n\times n}|trA=0\}$
	\item $\mathfrak{gl_n(\mathbb{R})}\cong M_{n\times n}$
	\item $\mathfrak{u_n(\mathbb{R})}=\{X\in M_{n\times n}|-X=X^+\}$
	\item $\mathfrak{o_n(\mathbb{R})}=\{X\in M_{n\times n}|-X=X^T\}$
\end{itemize}
\end{frame}
\begin{frame} {Teoria della rappresentazione}
	\begin{Def}
		Rappresentazione di gruppo: è un omomorfismo di gruppi $\rho:G\rightarrow GL(V)$. Ovvero $\rho(gh)=\rho(g)\circ\rho(h)$ per ogni $g,h\in G$.
		Rappresentazione di un'algebra di lie: è un morfismo di algebre di Lie $\pi:\mathfrak{g}\rightarrow End(V)$. Ovvero $\pi([X,Y])=[\pi(X),\pi(Y)]$.
	\end{Def}
Rappresentazione irriducibile: $\{0\}, V$ unici sottospazi invarianti.\\
Se $G$ è gruppo di Lie di matrici e $\rho$ è rappresentazione, esiste un'unica rappresentazione indotta $d\rho$ di $\mathfrak{g}$.
\begin{alertblock}{Proposizione}
	Se $G$ è un gruppo di Lie di matrici connesso allora una $\rho$ è irriducibile se e solo se $d\rho$ lo è.  
\end{alertblock}
\end{frame}
\begin{frame} {I gruppi $SU_2(\mathbb{C})$ e $SO_3(\mathbb{R})$}
	\begin{itemize}
		\item $SU_2(\mathbb{C})$ è semplicemente connesso compatto e semplicemente connesso;
		\item $SO_3(\mathbb{R})$ è connesso ma non semplicemente connesso;
		\item Esiste una mappa 2:1 tra $SU_2(\mathbb{C})$ e $SO_3(\mathbb{R})$: il gruppo $SU_2(\mathbb{C})$ copre due volte $SO_3(\mathbb{R})$ ed è un rivestimento universale.
	\end{itemize}
\begin{alertblock}{}
	Le rappresentazioni di $SO_3(\mathbb{R})$ sono quelle di $SU_2(\mathbb{C})$ dove $\pm\mathbb{I}$ agisce come l'identità.
\end{alertblock}
\end{frame}
\begin{frame} {I gruppi $SU_2(\mathbb{C})$ e $SO_3(\mathbb{R})$}
	Per ogni $d\rho$ irriducibile di $\mathfrak{su_2(\mathbb{C})_c}$, esiste $m\in \mathbb{N}$ e $u_0,...,u_m$ tali che:
	\begin{itemize}
		\item $d\rho(H)u_k=(m-2k)u_k$;
		\item $d\rho(X)u_k=0;$ se k=0 e $d\rho(X)u_k=k(m-(k-1))u_{k-1}$ se $k>0$;
		\item $d\rho(Y)u_k=u_{k+1}$ se $k<m$ e $d\rho(Y)u_k=0$ se $k=m$
	\end{itemize}
	\begin{alertblock}{}
		Per ogni $m$ intero pari esiste una rappresentazione irriducibile di $SO_3(\mathbb{R})$ di dimensione $m+1$ dispari.
	\end{alertblock}
\end{frame}
\begin{frame}{Quantizzazione del momento angolare}
	
\end{frame}
\end{document}
